% ---
% Capa
% ---
\imprimircapa
% ---

% ---
% Folha de rosto
% (o * indica que haverá a ficha bibliográfica)
% ---
\imprimirfolhaderosto*
% ---

% ---
% Inserir a ficha bibliografica
% ---
% http://ficha.bu.ufsc.br/
\begin{fichacatalografica}
	\includepdf{beforetext/ficha-catalografica-tcc.pdf}
\end{fichacatalografica}
% ---

\setlength{\ABNTEXsignwidth}{10cm}

% ---
% Inserir folha de aprovação
% ---
\begin{folhadeaprovacao}
	\OnehalfSpacing
	\centering
	\imprimirautor\\%
	\vspace*{10pt}		
	\textbf{\imprimirtitulo}%
	\ifnotempty{\imprimirsubtitulo}{:~\imprimirsubtitulo}\\%
	%		\vspace*{31.5pt}%3\baselineskip
	\vspace*{\baselineskip}
	%\begin{minipage}{\textwidth}
	% ~do~\imprimirprograma~do~\imprimircentro~da~\imprimirinstituicao~para~a~obtenção~do~título~de~\imprimirformacao.
	Este~\imprimirtipotrabalho~foi julgado adequado para obtenção do título de \imprimirformacao~e aprovado em sua forma final pela banca examinadora. \\
		\vspace*{\baselineskip}
	\imprimirlocal, \imprimirdata. \\
	\vspace*{2\baselineskip}
	\assinatura{\OnehalfSpacing\imprimircoordenador \\ \imprimircoordenadorRotulo~do Curso}
	\vspace*{2\baselineskip}
	\textbf{Banca Examinadora:} \\
	\vspace*{\baselineskip}
	\assinatura{\OnehalfSpacing\imprimirorientador \\ Presidente da Banca}
	%\end{minipage}%
	\vspace*{\baselineskip}
	\assinatura{Prof. X Y Z, Me.\\
	Avaliador \\
	\imprimirinstituicao}

	\vspace*{\baselineskip}
	\assinatura{Prof. X Y Z, Dr.\\
	Avaliador \\
	\imprimirinstituicao}


\end{folhadeaprovacao}
% ---

% ---
% Dedicatória
% ---
%\begin{dedicatoria}
%	\vspace*{\fill}
%	\noindent
%	\begin{adjustwidth*}{}{5.5cm}     
%		Este trabalho é dedicado aos meus colegas de classe e aos meus queridos pais.
%	\end{adjustwidth*}
%\end{dedicatoria}
% ---

% ---
% Agradecimentos
% ---
\begin{agradecimentos}

Agradeço a meu pai, minha mãe, meu cachorro, minha sogra e por último e menos importante, meu orientador.
\end{agradecimentos}
% ---

% ---
% Epígrafe
% ---
%\begin{epigrafe}
%	\vspace*{\fill}
%	\begin{flushright}
%		\textit{``Texto da Epígrafe.\\
%			Citação relativa ao tema do trabalho.\\
%			É opcional. A epígrafe pode também aparecer\\
%			na abertura de cada seção ou capítulo.\\
%			Deve ser elaborada de acordo com a NBR 10520.''\\
%			(Autor da epígrafe, ano)}
%	\end{flushright}
%\end{epigrafe}
% ---

% ---
% RESUMOS
% ---

% resumo em português
\setlength{\absparsep}{18pt} % ajusta o espaçamento dos parágrafos do resumo
\begin{resumo}
	\SingleSpacing

O presente estudo teve como objetivo principal comparar diferentes estratégias de construção de \textit{ensembles} heterogêneos aplicados à \textit{análise de sentimentos} em comentários de filmes. Para isso, foram utilizadas duas bases distintas: uma coletada do \textit{Letterboxd}, contendo mais de 100 mil avaliações rotuladas automaticamente, e a base clássica do \textit{IMDb}. Cada conjunto foi avaliado em dois cenários: com e sem \textit{pré-processamento textual}. Dez modelos de classificação foram treinados e tiveram seus hiperparâmetros otimizados com o \textit{framework Optuna}, utilizando \textit{validação cruzada}, formando um \textit{pool} inicial de classificadores. A partir desse conjunto, três estratégias de combinação foram analisadas: \textit{Sem Seleção} (\textit{SS}), \textit{Seleção por Acurácia} (\textit{SA}) e \textit{Seleção por Acurácia e Diversidade} (\textit{SAD}).Os resultados demonstraram diferenças consistentes entre as estratégias, evidenciando que a \textit{diversidade} entre os modelos é um fator determinante para melhorar o desempenho dos \textit{ensembles}. O \textit{Stacking}, em particular, mostrou-se mais eficaz na exploração dessa diversidade. Conclui-se que \textit{ensembles} heterogêneos construídos com critérios robustos de seleção (\textit{SA} e \textit{SAD}) apresentam desempenho superior e maior estabilidade na tarefa de análise de sentimentos, sendo a escolha ideal dependente do equilíbrio desejado entre performance e custo computacional.


	\textbf{Palavras-chave}: Análise de Sentimentos; Ensemble; Seleção; Optuna; Validação Cruzada
\end{resumo}

% resumo em inglês
\begin{resumo}[Abstract]
	\SingleSpacing
	\begin{otherlanguage*}{english}


The present study aimed to compare different strategies for constructing heterogeneous \textit{ensembles} applied to \textit{sentiment analysis} in movie reviews. For this purpose, two distinct datasets were used: one collected from \textit{Letterboxd}, containing over 100 thousand automatically labeled reviews, and the classic \textit{IMDb} dataset. Each set was evaluated in two scenarios: with and without \textit{text preprocessing}. Ten classification models were trained, and their hyperparameters were optimized with the \textit{Optuna framework} using \textit{cross-validation}, forming an initial \textit{pool} of classifiers. From this set, three combination strategies were analyzed: \textit{No Selection} (\textit{SS}), \textit{Selection by Accuracy} (\textit{SA}), and \textit{Selection by Accuracy and Diversity} (\textit{SAD}).
The results demonstrated consistent differences among the strategies, showing that \textit{diversity} among models is a determining factor for improving \textit{ensemble} performance. \textit{Stacking}, in particular, proved more effective in leveraging this diversity. It is concluded that heterogeneous \textit{ensembles} built with robust selection criteria (\textit{SA} and \textit{SAD}) exhibit superior performance and greater stability in the sentiment analysis task, with the ideal choice depending on the desired balance between performance and computational cost.

\textbf{Keywords}: Sentiment Analysis; Ensemble; Selection; Optuna; Cross-Validation
	\end{otherlanguage*}
\end{resumo}

%% resumo em francês 
%\begin{resumo}[Résumé]
% \begin{otherlanguage*}{french}
%    Il s'agit d'un résumé en français.
% 
%   \textbf{Mots-clés}: latex. abntex. publication de textes.
% \end{otherlanguage*}
%\end{resumo}
%
%% resumo em espanhol
%\begin{resumo}[Resumen]
% \begin{otherlanguage*}{spanish}
%   Este es el resumen en español.
%  
%   \textbf{Palabras clave}: latex. abntex. publicación de textos.
% \end{otherlanguage*}
%\end{resumo}
%% ---

{%hidelinks
	\hypersetup{hidelinks}
	% ---
	% inserir lista de ilustrações
	% ---
	\pdfbookmark[0]{\listfigurename}{lof}
	\listoffigures*
	\cleardoublepage
	% ---
	
	% ---
	% inserir lista de quadros
	% ---
	\pdfbookmark[0]{\listofquadrosname}{loq}
	\listofquadros*
	\cleardoublepage
	% ---
	
	% ---
	% inserir lista de tabelas
	% ---
	\pdfbookmark[0]{\listtablename}{lot}
	\listoftables*
	\cleardoublepage
	% ---
	
	% ---
	% inserir lista de abreviaturas e siglas (devem ser declarados no preambulo)
	% ---
	\imprimirlistadesiglas
	% ---
	
	% ---
	% inserir lista de símbolos (devem ser declarados no preambulo)
	% ---
	\imprimirlistadesimbolos
	% ---
	
	% ---
	% inserir o sumario
	% ---
	\pdfbookmark[0]{\contentsname}{toc}
	\tableofcontents*
	\cleardoublepage
	
}%hidelinks
% ---
