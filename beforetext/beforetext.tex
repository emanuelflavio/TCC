% ---
% Capa
% ---
\imprimircapa
% ---

% ---
% Folha de rosto
% (o * indica que haverá a ficha bibliográfica)
% ---
\imprimirfolhaderosto*
% ---

% ---
% Inserir a ficha bibliografica
% ---
% http://ficha.bu.ufsc.br/
\begin{fichacatalografica}
	\includepdf{beforetext/ficha-catalografica-tcc.pdf}
\end{fichacatalografica}
% ---

\setlength{\ABNTEXsignwidth}{10cm}

% ---
% Inserir folha de aprovação
% ---
%\begin{folhadeaprovacao}
%	\OnehalfSpacing
%	\centering
%	\imprimirautor\\%
%	\vspace*{10pt}		
%	\textbf{\imprimirtitulo}%
%	\ifnotempty{\imprimirsubtitulo}{:~\imprimirsubtitulo}\\%
%	%		\vspace*{31.5pt}%3\baselineskip
%	\vspace*{\baselineskip}
%	%\begin{minipage}{\textwidth}
%	% ~do~\imprimirprograma~do~\imprimircentro~da~\imprimirinstituicao~para~a~obtenção~do~título~de~\imprimirformacao.
%	Este~\imprimirtipotrabalho~foi julgado adequado para obtenção do título de \imprimirformacao~e aprovado em sua forma final pela banca examinadora. \\
%		\vspace*{\baselineskip}
%	\imprimirlocal, \imprimirdata. \\
%	\vspace*{2\baselineskip}
%	\assinatura{\OnehalfSpacing\imprimircoordenador \\ \imprimircoordenadorRotulo~do Curso}
%	\vspace*{2\baselineskip}
%	\textbf{Banca Examinadora:} \\
%	\vspace*{\baselineskip}
%	\assinatura{\OnehalfSpacing\imprimirorientador \\ Presidente da Banca}
%	%\end{minipage}%
%	\vspace*{\baselineskip}
%	\assinatura{Prof. Dr. Rodrigo Yoshio Tamae\\
%	Avaliador \\
%	\imprimirinstituicao}
%
%	\vspace*{\baselineskip}
%	\assinatura{Prof. Dr. Ricardo Azevedo Moreira Da Silva \\
%	Avaliador \\
%	\imprimirinstituicao}
%
%
%\end{folhadeaprovacao}

\includepdf[pages=1]{folha1.pdf}

% ---

% ---
% Dedicatória
% ---
%\begin{dedicatoria}
%	\vspace*{\fill}
%	\noindent
%	\begin{adjustwidth*}{}{5.5cm}     
%		Este trabalho é dedicado aos meus colegas de classe e aos meus queridos pais.
%	\end{adjustwidth*}
%\end{dedicatoria}
% ---

% ---
% Agradecimentos
% ---
\begin{agradecimentos}

Agradeço primeiramente a Deus e a Nossa Senhora, por sempre estarem ao meu lado, iluminarem meu caminho e fortalecerem minha fé. Cada passo que dei foi abençoado, e por isso consegui chegar até aqui.

A meus pais e à minha família, pelo apoio incondicional, por serem minha base e meu conforto, e por me darem o abrigo de seu abraço mesmo quando errei e tive dias difíceis. Deixo a eles minha mais profunda gratidão.

À Alana, minha namorada, sou grato pelo amor, pelo cuidado, pelas palavras de incentivo e por estar ao meu lado em cada etapa. Seu apoio fez toda a diferença e tornou esta caminhada muito mais leve e significativa.

Aos meus amigos, obrigado por estarem comigo, me encorajando e dando força sempre que precisei.

À minha orientadora,  Profª. Me. Débora da Conceição Araújo, meus agradecimentos pela dedicação, profissionalismo, paciência e pela orientação prestada ao desenvolvimento deste trabalho.

A todos que de alguma forma colaboraram para a finalização deste trabalho, deixo aqui meu reconhecimento e gratidão.  

\end{agradecimentos}
% ---

% ---
% Epígrafe
% ---
%\begin{epigrafe}
%	\vspace*{\fill}
%	\begin{flushright}
%		\textit{``Texto da Epígrafe.\\
%			Citação relativa ao tema do trabalho.\\
%			É opcional. A epígrafe pode também aparecer\\
%			na abertura de cada seção ou capítulo.\\
%			Deve ser elaborada de acordo com a NBR 10520.''\\
%			(Autor da epígrafe, ano)}
%	\end{flushright}
%\end{epigrafe}
% ---

% ---
% RESUMOS
% ---

% resumo em português
\setlength{\absparsep}{18pt} % ajusta o espaçamento dos parágrafos do resumo
\begin{resumo}
	\SingleSpacing


O presente estudo teve como objetivo principal comparar diferentes estratégias de construção de \textit{ensembles} heterogêneos aplicados à análise de sentimento em comentários de filmes. Para isso, foram utilizadas duas bases de dados distintas: uma coletada do \textit{Letterboxd}, construída por meio de um \textit{crawler}, contendo mais de 100 mil avaliações rotuladas com base na avaliação dos comentários, onde notas de 0 a 2,5 eram negativas e de 3 a 5 positivas; e a base clássica do \textit{IMDb}.
Cada conjunto de dados foi avaliado em dois cenários: com e sem pré-processamento textual. Dez modelos de classificação foram treinados e tiveram seus hiperparâmetros otimizados com o \textit{framework Optuna}, utilizando validação cruzada, formando um \textit{pool} inicial de classificadores.
A partir desse conjunto, três estratégias de combinação foram analisadas: Sem Seleção (SS), Seleção por Acurácia (SA) e Seleção por Acurácia e Diversidade (SAD).
Os resultados demonstraram diferenças consistentes entre as estratégias, evidenciando que a diversidade entre os modelos é um fator determinante para melhorar o desempenho dos \textit{ensembles}. Um dos métodos de aprendizado de máquina \textit{ensemble}, o Stacking, em particular, mostrou-se mais eficaz na exploração dessa diversidade.
Conclui-se que \textit{ensembles} heterogêneos construídos com critérios robustos de seleção (SA e SAD) apresentam desempenho superior e maior estabilidade na tarefa de análise de sentimento. A escolha ideal, no entanto, depende do equilíbrio desejado entre performance e custo computacional.



	\textbf{Palavras-chave}: Análise de Sentimentos; Ensemble de Classificadores; Otimização de Hiperparâmetros
\end{resumo}

% resumo em inglês
\begin{resumo}[Abstract]
	\SingleSpacing
	\begin{otherlanguage*}{english}


The main objective of the present study was to compare different heterogeneous \textit{ensemble} construction strategies applied to sentiment analysis in movie reviews. For this purpose, two distinct databases were used: one collected from \textit{Letterboxd}, built using a \textit{crawler}, containing over 100,000 reviews labeled based on the review score, where ratings from 0 to 2.5 were considered negative and 3 to 5 were positive; and the classic \textit{IMDb} database.
Each dataset was evaluated in two scenarios: with and without textual preprocessing. Ten classification models were trained and had their hyperparameters optimized with the \textit{Optuna framework}, using cross-validation, forming an initial classifier \textit{pool}.
From this set, three combination strategies were analyzed: No Selection (NS), Selection by Accuracy (SA), and Selection by Accuracy and Diversity (SAD).
The results demonstrated consistent differences between the strategies, evidencing that the diversity among models is a determining factor for improving \textit{ensemble} performance. One of the \textit{ensemble} machine learning methods, Stacking, in particular, proved more effective in exploring this diversity.
It is concluded that heterogeneous \textit{ensembles} constructed with robust selection criteria (SA and SAD) show superior performance and greater stability in the sentiment analysis task. The ideal choice, however, depends on the desired balance between performance and computational cost.

\textbf{Keywords}: Sentiment Analysis; Classifier Ensemble; Hyperparameter Optimization
	\end{otherlanguage*}
\end{resumo}

%% resumo em francês 
%\begin{resumo}[Résumé]
% \begin{otherlanguage*}{french}
%    Il s'agit d'un résumé en français.
% 
%   \textbf{Mots-clés}: latex. abntex. publication de textes.
% \end{otherlanguage*}
%\end{resumo}
%
%% resumo em espanhol
%\begin{resumo}[Resumen]
% \begin{otherlanguage*}{spanish}
%   Este es el resumen en español.
%  
%   \textbf{Palabras clave}: latex. abntex. publicación de textos.
% \end{otherlanguage*}
%\end{resumo}
%% ---

{%hidelinks
	\hypersetup{hidelinks}
	% ---
	% inserir lista de ilustrações
	% ---
	\pdfbookmark[0]{\listfigurename}{lof}
	\listoffigures*
	\cleardoublepage
	% ---
	
	
	% ---
	% inserir lista de tabelas
	% ---
	\pdfbookmark[0]{\listtablename}{lot}
	\listoftables*
	\cleardoublepage
	% ---
	

	
	
	% ---
	% inserir o sumario
	% ---
	\pdfbookmark[0]{\contentsname}{toc}
	\tableofcontents*
	\cleardoublepage
	
}%hidelinks
% ---
