
\chapter{Trabalhos Relacionados}\label{fundamentacao}

Existem diversas pesquisas relevantes nas áreas de Análise de Sentimentos, Modelos de Aprendizado de Máquina, \textit{Ensemble Learning} e Processamento de Linguagem Natural, que contribuem significativamente para o avanço desses campos.


No trabalho de \textcite{kazmaier2022power}, investiga-se como a aprendizagem por conjunto (\textit{ensemble learning}) pode contribuir para a análise de sentimentos. Os autores destacam que, embora haja um crescente interesse por técnicas de ensemble learning na comunidade de aprendizado de máquina, seu uso específico na classificação de sentimentos ainda é limitado. Observa-se, ainda, que grande parte das pesquisas concentra-se em ensembles homogêneos, embora os ensembles heterogêneos possam ser bastante eficazes ao combinar diferentes modelos. O artigo propõe uma nova abordagem para seleção de modelos que compõem o ensemble, evitando o armazenamento das previsões individuais e o retreinamento custoso de todos os modelos candidatos. Utilizando quatro conjuntos de dados de análise de sentimentos, os autores observaram uma melhoria de desempenho de até 5,53\% em relação ao melhor modelo individual.


Em \textcite{sirqueira2024evaluation}, é proposta uma avaliação de modelos de ensemble utilizando modelos Transformers para a tarefa de Reconhecimento de Entidades Nomeadas (Named Entity Recognition – NER) em textos públicos brasileiros. O estudo destaca os avanços significativos no Processamento de Linguagem Natural, impulsionados principalmente pelo desenvolvimento de modelos de aprendizado profundo baseados em Transformers. A análise de dados abertos no contexto brasileiro, como documentos publicados no Diário Oficial da União, é considerada crucial para a transparência e o acesso à informação. Os autores testaram um conjunto de modelos baseados em variações do BERT, combinando diferentes estratégias de ensemble, e alcançaram melhorias de até 11\% no corpus proposto, em comparação com abordagens clássicas de NER baseadas apenas no BERT.


No trabalho de \textcite{santos2023analysis}, é investigada a classificação de sentimentos em uma amostra de textos publicados no Twitter, em português, sobre as eleições presidenciais brasileiras de 2022. Os autores destacam que o crescimento da internet e das redes sociais facilitou o acesso a informações sobre a opinião pública, mas que a análise manual de grandes volumes de comentários se torna inviável, exigindo o uso de tecnologias. Para isso, foi utilizado o processo de Descoberta de Conhecimento em Banco de Dados, aliado a técnicas de aprendizado de máquina, para analisar e classificar os tweets em opiniões positivas, neutras e negativas. Foram empregadas duas representações textuais clássicas (Bag of Words e TF-IDF) e seis classificadores: Naive Bayes, Árvore de Decisão, Random Forest, K-Nearest Neighbors, MLP e SVM. Os resultados, com base em um conjunto de dados balanceado, indicaram que Jair Bolsonaro apresentou a maior proporção de sentimentos positivos, Luiz Inácio Lula da Silva a maior de sentimentos neutros, e Ciro Gomes a maior de sentimentos negativos.

\textcite{inbook} investigaram o impacto da subjetividade do texto na analise de sentimentos, comparando diferentes técnicas de representação textual, sendo elas TF-IDF, Word2Vec e BERT. O estudo foi conduzido em duas etapas: primeiro os autores realizaram a classificação da subjetividade, separando os textos subjetivos dos objetivos; depois, aplicaram a classificação de sentimentos apenas nos textos subjetivos, logo que os objetivos não expressam opinião. Os resultados mostraram que o BERT obteve desempenho superior, com 99,77\% de acurácia na classificação da subjetividade, e metodo do TF-IDF apresentando melhor desempenho com 91,29\% de acuracia na classificação de sentimentos, superando o Word2Vec e BERT e mostrando que representações mais simples ainda são competitivas.

Já em \textcite{ribeiro2016sentibench}, foi proposto um banchmark para a análise de sentimento, comparando 24 métodos que são amplamente utiliados na literatura. Foi feita a avaliação em 18 bases de dados rotuladas, abrangendo redes sociais, avaliação de filmes, produtos e comentários de notícias. Os resultados mostraram que não existe um método que fosse universalmente superior. Os desempenhos variavam de acordo com o tipo de dado e domínio. Além disso, pode-se observar que a maioria das abordagens apresentou viés positivo, classificando melhor os textos positivos que os negativos. Os estudos mostraram a importância de utilizar diversos métodos e bases de dados.  

No trabalho de \textcite{kilimci2018deep} foi investigado o uso de ensembles heterogeneos, combinando classificadores tradicionais com modelos de Deep Learning, com word embeddings pré-treinados para representação textual. O estudo avaliou diferentes formas de representação textual como TF-IDF, media de embeddings e combinações hibridas. Foram utlizadas 8 bases de dados e os resultados mostraram que osensembles heterogêneos apresentaram desempenho superior aos classificadores individuais em todos os cenarios analisados. Em especial a combinação de deep learning com word embeddings, que teve ganhos consistentes de acurácia, e mostrando que a diversidade é um fator para a melhora de desmpenho em tarefas de classificação textual. 


O presente trabalho, em relação aos estudos de \textcite{kazmaier2022power}, \textcite{sirqueira2024evaluation}, \textcite{santos2023analysis}, \textcite{inbook}, \textcite{ribeiro2016sentibench} e \textcite{kilimci2018deep}, apresenta contribuições metodologicas ao realizar uma avaliação sistemática e abrangente de ensembles heterogêneos aplicados à análise de sentimentos. Diferentemente de trabalhos que focam exclusivamente em ensembles homogêneos, em modelos Transformers de alto custo computacional ou em classificadores individuais, esta pesquisa investiga três estratégias distintas de seleção de modelos, considerando simultaneamente desempenho preditivo e diversidade, por meio do uso do $Q$-statistic. Além disso, são empregados dez modelos base de menor custo computacional, otimizados com o framework Optuna, permitindo uma análise comparativa entre cenários com e sem pré-processamento textual. O trabalho também aplica duas técnicas de ensemble e avalia duas bases de dados de comentários de filmes, incluindo uma base inédita do Letterboxd construída especificamente para esta pesquisa, ampliando a diversidade experimental. Esses elementos contribuem para tornar o estudo mais completo, replicável e alinhado ao estado da arte, reforçando a relevância do uso de ensembles heterogêneos eficientes em tarefas de análise de sentimentos.
